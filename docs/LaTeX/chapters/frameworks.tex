Egy szimulációskörnyezet szükséges és előnyös rajintelligenciás osztott algoritmusok kidolgozásához. Jóval lerövidíti a fejlesztési időt,
mivel egyrészt nem szükséges számos robot megvásárlása, másrészt a robotok felprogramozása összemérhetetlenül gyorsabban történik.

\subsection{Robot szimulátorok adta lehetőségek tanulmányozása} 

A robotszimulációs környezet világában két nyílt forráskódu, elterjedtebb szoftver között lehet választani: az ARGoS3 és a V-REP.
A következőkben szemléltetve lesz a két szimulációs környezet előnyei, hátrányai.

\textbf{V-REP} \url{http://www.coppeliarobotics.com/}
\begin{itemize}
    \item Támogatott nyelvek robot kontrollerlogikájának megírására: C/C++, Python, Java, Lua, Matlab, Octave
    \item Rendkívül valósághű
    \item Ingyenes, van fizetős verzió is
    \item Robotrajokra nincs optimalizálva
    \item Aktívan fejlesztik, jelennek meg új verziók
    \item Nagy a felhasználók száma
    \item Nyílt forráskódu
\end{itemize}

\textbf{ARGoS3} \url{https://www.argos-sim.info/index.php}
\begin{itemize}
    \item Támogatott nyelvek robot kontrollerlogikájának megírására: C/C++ és Lua
    \item Nagyobb az absztrakciós szint, nem annyira valósághű
    \item Robotrajokra teljesen optimalizálva van
    \item Ingyenes
    \item Fejlesztés alatt áll, bár az új verziók nem konszisztensen érkeznek (2019 júliusán jött ki új verzió, az ezelőtti verzió 2016-os)
    \item Viszonylag kevesen használják, nehezebb a hibákra megoldást találni
    \item Nyílt forráskódu
\end{itemize}

A két robotszimulátor között a választás az ARGoS-ra esett, mivel egy nagyobb absztrakciós szintet ajánl, ezért könnyebb magára az algoritmusra, a logikára fektetni a hangsúlyt. 
Ez a nagyobb absztrakciós szint ahhoz is vezet, hogy egy sok robotot tartalmazó robotraj esetén a számítási kapacitás közel se olyan sok, mint a V-REP-nél. 



