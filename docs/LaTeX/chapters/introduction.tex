A rajintelligencia alapötlete, hogy számos limitált számításikapacitással rendelkező agensek képesek egy komplex feladat megoldására.
Az ötlet az állatvilágban megvizsgált szószerinti rajok tanulmányozásából született. Rajokba szerveződnek a hangyák, méhek, madarak, halak és sok más állatfajta.
A méhek esetén kimutatható, hogy egy olyan komplex feladat megoldásában, mint például a kaptár optimális helyének a kiválasztásában, az esetek
hozzávetőlegesen 80 százalékában optimális megoldásra jutnak.
Az állatvilágot utánozva jelentek meg ennek az alapötletnek számos alkalmazásai nem csak robotikában, hanem a számítástechnika számos ágazatában (TODO: példák sorolása).
E ötlet követése által lehetséges olyan osztott algoritmusok kidolgozása, amelyek implementálása által sok limitált számításikapacítással rendelkező
agens egy komplex feladat megoldására képes.
