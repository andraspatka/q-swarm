\documentclass[12pt, twoside]{report}
\usepackage[utf8]{inputenc}
\usepackage{graphicx}
\graphicspath{ {figures/} }
\usepackage[a4paper,width=150mm,top=25mm,bottom=25mm,bindingoffset=6mm]{geometry}

\usepackage[magyar]{babel}
\usepackage{t1enc}

\usepackage{color}
\definecolor{mygreen}{rgb}{0,0.6,0}
\definecolor{mygray}{rgb}{0.5,0.5,0.5}
\definecolor{mymauve}{rgb}{0.58,0,0.82}

\usepackage{listings}
\usepackage{xcolor}
\lstset { %
    language=C++,
    backgroundcolor=\color{black!5}, % set backgroundcolor
    basicstyle=\footnotesize\sffamily\ttfamily , % basic font setting
  breaklines=true,                 % automatic line breaking only at whitespace
  captionpos=b,                    % sets the caption-position to bottom
  commentstyle=\color{mygreen},    % comment style
  %keywordstyle=\color{blue},       % keyword style
  stringstyle=\color{mymauve}     % string literal style
}


\usepackage{xcolor}
\usepackage{hyperref}
\hypersetup{
    colorlinks,
    linkcolor=black,
    citecolor=black,
    urlcolor={blue!80!black}
}
% 
\urlstyle{same}

%pseudocode package-ek
\usepackage{amsmath}
\usepackage{algorithm}
\usepackage[noend]{algpseudocode}

%definition of tab command
\newcommand\tab[1][1cm]{\hspace*{#1}}

\title{
	{Szimulált robot raj vezérlése}\\
	{\large Sapientia\\
	Erdélyi Magyar Tudományegyetem, Marosvásárhely}
}
\author{Patka Zsolt-András}
\date{2020}

%%%%%%%%%%%%%%%%%%%%%%%%%%%%%%%%%%%%%%%%%%%%%%%%%%%%%%%%%%%%%%%%%%%%%%%%%
\begin{document}

\maketitle
\pagenumbering{gobble}
\clearpage\mbox{}\clearpage

\chapter*{Kivonat}
\input{chapters/abstractHu}

\chapter*{Abstract}
\input{chapters/abstractEn}

\chapter*{Extras}
\input{chapters/abstractRo}


\tableofcontents


\chapter{Bevezető}
\pagenumbering{arabic}
A rajintelligencia alapötlete, hogy számos limitált számításikapacitással rendelkező agensek képesek egy komplex feladat megoldására.
Az ötlet az állatvilágban megvizsgált szószerinti rajok tanulmányozásából született. Rajokba szerveződnek a hangyák, méhek, madarak, halak és sok más állatfajta.
A méhek esetén kimutatható, hogy egy olyan komplex feladat megoldásában, mint például a kaptár optimális helyének a kiválasztásában, az esetek
hozzávetőlegesen 80 százalékában optimális megoldásra jutnak.
Az állatvilágot utánozva jelentek meg ennek az alapötletnek számos alkalmazásai nem csak robotikában, hanem a számítástechnika számos ágazatában (TODO: példák sorolása).
E ötlet követése által lehetséges olyan osztott algoritmusok kidolgozása, amelyek implementálása által sok limitált számításikapacítással rendelkező
agens egy komplex feladat megoldására képes.


\chapter{A dolgozat célja}
\input{chapters/projectAim}

\chapter{Szakirodalmi háttér bemutatása}
A Q-Learning egy modell nélküli megerősítési tanulásos algoritmus. Modell nélküli, tehát nem szükséges a környezetről
egy modellt készítenie. 

\chapter{Felhasznált Szoftverkeretrendszerek}
Egy szimulációskörnyezet szükséges és előnyös rajintelligenciás osztott algoritmusok kidolgozásához. Jóval lerövidíti a fejlesztési időt,
mivel egyrészt nem szükséges számos robot megvásárlása, másrészt a robotok felprogramozása összemérhetetlenül gyorsabban történik.

\subsection{Robot szimulátorok adta lehetőségek tanulmányozása} 

A robotszimulációs környezet világában két nyílt forráskódu, elterjedtebb szoftver között lehet választani: az ARGoS3 és a V-REP.
A következőkben szemléltetve lesz a két szimulációs környezet előnyei, hátrányai.

\textbf{V-REP} \url{http://www.coppeliarobotics.com/}
\begin{itemize}
    \item Támogatott nyelvek robot kontrollerlogikájának megírására: C/C++, Python, Java, Lua, Matlab, Octave
    \item Rendkívül valósághű
    \item Ingyenes, van fizetős verzió is
    \item Robotrajokra nincs optimalizálva
    \item Aktívan fejlesztik, jelennek meg új verziók
    \item Nagy a felhasználók száma
    \item Nyílt forráskódu
\end{itemize}

\textbf{ARGoS3} \url{https://www.argos-sim.info/index.php}
\begin{itemize}
    \item Támogatott nyelvek robot kontrollerlogikájának megírására: C/C++ és Lua
    \item Nagyobb az absztrakciós szint, nem annyira valósághű
    \item Robotrajokra teljesen optimalizálva van
    \item Ingyenes
    \item Fejlesztés alatt áll, bár az új verziók nem konszisztensen érkeznek (2019 júliusán jött ki új verzió, az ezelőtti verzió 2016-os)
    \item Viszonylag kevesen használják, nehezebb a hibákra megoldást találni
    \item Nyílt forráskódu
\end{itemize}

A két robotszimulátor között a választás az ARGoS-ra esett, mivel egy nagyobb absztrakciós szintet ajánl, ezért könnyebb magára az algoritmusra, a logikára fektetni a hangsúlyt. 
Ez a nagyobb absztrakciós szint ahhoz is vezet, hogy egy sok robotot tartalmazó robotraj esetén a számítási kapacitás közel se olyan sok, mint a V-REP-nél. 





%Terv
\chapter{Gyakorlati megvalósítás}
\section{Célkövetés és akadálykerülés a push-pull erők használatával}

A push-pull erők elv követésével felírhatóak azok az erők amik a cél fele mutatnak, mint húzó erők, és azok amik az akadály felé mutatnak, mint taszító erők.


\section{Célkövetés és akadálykerülés QLearning használatával}

\begin{table}[H]
	\begin{center}
		\caption{Jutalommatrix, Akkadálykerülő és célkövető agensnek}
		\begin{tabular}{l|c|c|c|c}
		\textbf{Állapot} & $STOP$     & $TURN\_LEFT$  & $TURN\_RIGHT$ & FORWARD \\
		\hline         
        FOLLOW           & -1         & 0             & 0             & 1  \\
        \hline         
        UTURN            & -1         & 0             & 0             & 0  \\
        \hline         
        OBST\_LEFT       & -1         & 0             & 0.1           & 0  \\
        \hline         
        OBST\_RIGHT      & -1         & 0.1           & 0             & 0  \\
        \hline         
        OBST\_FORWARD    & -1         & 0.1           & 0             & 0  \\
        \hline         
        WANDER           & -1         & 0             & 0             & 0.1  \\
        \hline         
        IDLE             & 1          & 0             & 0             & -1  \\
		\end{tabular}
	\end{center}
\end{table}


\subsection{}

\chapter{A rendszer tesztelése}
\input{chapters/results}

%következtetés
\chapter{Következtetések}
\input{chapters/summary}

\bibliographystyle{ieeetr}
\bibliography{References}

%appendix
\input{chapters/appendix}


\end{document}